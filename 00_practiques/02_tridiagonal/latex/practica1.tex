\documentclass[11pt,a4paper]{article}
\usepackage[utf8]{inputenc}
\usepackage[catalan]{babel}
\usepackage{amsmath}
\usepackage{amsfonts}
\usepackage{amssymb}
\usepackage{mathtools}
\usepackage{geometry}
\usepackage{listingsutf8}
\usepackage{listings}
\usepackage{xcolor}
\usepackage{graphicx}
\usepackage{wrapfig}
\usepackage{indentfirst}
\geometry{top=25mm}

\lstset{%language=
                inputencoding=utf8/latin1,
                basicstyle=\ttfamily,
                keywordstyle=\color{blue}\ttfamily,
                stringstyle=\color{red}\ttfamily,
                commentstyle=\color{green}\ttfamily,
                columns=fullflexible,
  frame=single,
  breaklines=true,
  postbreak=\mbox{\textcolor{red}{$\hookrightarrow$}\space},                
                morecomment=[l][\color{magenta}]{\#}
}

\author{Marta Llagostera i Adrià Vilanova\vspace{-2ex} }
\title{Pràctica voluntària - Descomposició matriu tridiagonal i tridiagonal amb cantonades\vspace{-2ex}}
\date{13 de març de 2018}

\setlength{\parskip}{1em}
\begin{document}
\maketitle

\section{Descomposició matriu tridiagonal}

\textbf{Output de la terminal:}

\lstinputlisting{../terminal_pdf_1.txt}

\section{Descomposició matriu tridiagonal amb cantonades}

\textbf{Output de la terminal:}

\lstinputlisting{../terminal_pdf_2.txt}

\pagebreak

\section{Representació visual de les matrius dels dos exercicis}

\subsection{Descomposició matriu tridiagonal}

\lstinputlisting{../moreinfo_1.txt}

\subsection{Descomposició matriu tridiagonal amb cantonades}

\lstinputlisting{../moreinfo_2.txt}

\end{document}